%------------------------
% Resume Template
% Author : Anubhav Singh
% Github : https://github.com/xprilion
% License : MIT
%------------------------

\documentclass[letter,20pt]{article}
% PREAMBLE {{{
\usepackage{latexsym}
\usepackage[empty]{fullpage}
\usepackage{titlesec}
\usepackage{marvosym}
\usepackage[usenames,dvipsnames]{color}
\usepackage{verbatim}
\usepackage{enumitem}
\usepackage[pdftex]{hyperref}
\usepackage{fancyhdr}

\pagestyle{fancy}
\fancyhf{} % clear all header and footer fields
\fancyfoot{}
\renewcommand{\headrulewidth}{0pt}
\renewcommand{\footrulewidth}{0pt}

% Adjust margins
\addtolength{\oddsidemargin}{-0.530in}
\addtolength{\evensidemargin}{-0.375in}
\addtolength{\textwidth}{1in}
\addtolength{\topmargin}{-.45in}
\addtolength{\textheight}{1in}

\urlstyle{rm}

\raggedbottom
\raggedright
\setlength{\tabcolsep}{0in}

% Sections formatting
\titleformat{\section}{
  \vspace{-10pt}\scshape\raggedright\large
}{}{0em}{}[\color{black}\titlerule \vspace{-6pt}]

%-------------------------
% Custom commands
\newcommand{\resumeItem}[2]{
  \item\small{
    \textbf{#1}{: #2 \vspace{-2pt}}
  }
}

\newcommand{\resumeItemWithoutTitle}[1]{
  \item\small{
    {\vspace{-2pt}}
  }
}

\newcommand{\resumeSubheading}[4]{
  \vspace{-1pt}\item
    \begin{tabular*}{0.97\textwidth}{l@{\extracolsep{\fill}}r}
      \textbf{#1} & #2 \\
      \textit{#3} & \textit{#4} \\
    \end{tabular*}\vspace{-5pt}
}


\newcommand{\resumeSubItem}[2]{\resumeItem{#1}{#2}\vspace{-3pt}}

\renewcommand{\labelitemii}{$\circ$}

\newcommand{\resumeSubHeadingListStart}{\begin{itemize}[leftmargin=*]}
\newcommand{\resumeSubHeadingListEnd}{\end{itemize}}
\newcommand{\resumeItemListStart}{\begin{itemize}}
\newcommand{\resumeItemListEnd}{\end{itemize}\vspace{-5pt}}
% }}}

\begin{document}
% HEADER {{{
\begin{tabular*}{\textwidth}{l@{\extracolsep{\fill}}c@{\extracolsep{\fill}}r}
  \textbf{{\LARGE Carlos Vigil Vásquez}} \\
  \\
  Email: \href{mailto:cvigil2@uc.cl}{cvigil2@uc.cl} & Teléfono: +569-95644768 & \href{https://github.com/cvigilv}{Github: github.com/cvigilv}
\end{tabular*}
% }}}
% EDUCACION {{{
\section{~~Educación}
  \resumeSubHeadingListStart
    \resumeSubheading
		{Colegio Everest}{Santiago, Chile}
		{Educación media científico-humanista completa;}{Marzo 2011 - Diciembre 2014}
    \resumeSubheading
		{Pontificia Universidad Católica de Chile}{Santiago, Chile}
		{Licenciatura en Bioquímica;}{Marzo 2015 - Septiembre 2020}
    \resumeSubheading
		{Pontificia Universidad Católica de Chile}{Santiago, Chile}
		{Alumno en vías de titulación en Bioquímica;}{Septiembre 2020 - presente}
    \resumeSubHeadingListEnd
\vspace{-5pt}
% }}}
% HABILIDADES {{{
\section{~~Habilidades}
	\resumeSubHeadingListStart
	\resumeSubItem{Lenguajes humanos}{~~~~~~~~~~Castellano (nativo), Inglés (TOEFL score 101 of 120)}
	\resumeSubItem{Lenguajes de máquina}{~~~~~~Julia, Python, LaTeX, R, C++, SQL, Bash}
	\resumeSubItem{\textit{Frameworks}}{~~~~~~~~~~~~~~~~~~~~~~Graphs.jl, CUDA.jl, Plots.jl, Scikit-Learn, Pandas, NumPy, Matplotlib, Seaborn, NetworkX, Pingouin, dyplr, ggplot}
	\resumeSubItem{Herramientas}{~~~~~~~~~~~~~~~~~~~Git, GitHub, MySQL, SQLite}
	\resumeSubItem{Plataformas}{~~~~~~~~~~~~~~~~~~~~~~Linux, MacOS}
	\resumeSubItem{Conocimientos prácticos}{~~~Sistemas de recomendación, métodos predictivos, estadística, programación lineal, teoría de grafos / redes, diseño experimental, visualización de datos}
\resumeSubHeadingListEnd
\vspace{-5pt}
% }}}
% EXPERIENCIA {{{
\section{~~Experiencia}
  \resumeSubHeadingListStart
    \resumeSubheading{Bioestadística, Pontificia Universidad Católica de Chile}{Presencial}{Ayudante de curso}{Julio 2018 - Diciembre 2018}
    \resumeSubheading{Laboratorio de Diseño Molecular, Pontificia Universidad Católica de Chile}{Presencial / Remoto}{Asistente de investigación}{Julio 2017 - presente}
    \resumeSubheading{Laboratorio de Psicofisiología, Pontificia Universidad Católica de Chile}{Remoto}{Asistente de investigación}{Enero 2022 - presente}
\resumeSubHeadingListEnd
\vspace{-5pt}
% }}}
% PROYECTOS {{{
\section{~~Proyectos}
\resumeSubHeadingListStart
\resumeSubItem{SLiP: SchuellerLab Ligand Priorization Pipeline}{algoritmo de selección y priorización de candidatos químicos obtenidos a partir de un algoritmo predictivo desarrollado en el marco del Concurso de Investigación de Pregrado - Invierno 2017. Implementación de flujo de trabajo ayudó a acelerar en un orden de 10 veces el proceso de búsqueda y selección de candidatos a pruebas experimentales para compuestos químicos en un contexto de desarrollo farmacológico para enfermedades trombóticas.~~~~~~\textit{Tecnología usada:} GOLD, Python, Pandas, MatplotLib, Bash. \textit{Fecha de termino del proyecto:} Diciembre, 2017}
\vspace{-2pt}
\resumeSubItem{Integrando información química, estructural y relacional para predecir interacciones proteína-ligando utilizando redes moleculares }{algoritmo predictivo basado en redes heterogéneas para el descubrimiento y reposicionamiento de fármacos en un contexto de desarrollo y descubrimiento farmacológico desarrollado en el marco del curso BIO296F: ''Seminario de Investigación''. Desarrollo de método predictivo basado en redes mostró alcanzar y superar el rendimiento de métodos predictivos del estado del arte.~~~~~~~~\textit{Tecnología usada:} Python, Pandas, NetworkX, Matplotlib, Seaborn, Sklearn, SQL, Bash. \textit{Fecha de termino del proyecto:} Julio, 2020}
\vspace{-2pt}
\resumeSubItem{Paradoja de uniones débiles para la predicción \textit{de novo} de blancos (en desarrollo)}{desarrollo de algoritmo predictivo basado en el formalismo de inferencia basada en redes junto con nociones de similitud química y la paradoja de uniones débiles desarrollado en el marco del curso BIO285D ''Memoria de Investigación''. El algoritmo propuesto demostró presentar un rendimiento predictivo mejor que el visto en el estado de arte, además de presentar ventajas para la predicción de blancos en escenarios complejos.~~~~~~~~\textit{Tecnología usada:} Julia, Graphs.jl, CUDA.jl, Python, Pandas, Sklearn, Matplotlib, Seaborn, SQL, Bash. \textit{Fecha esperada para el termino del proyecto:} Diciembre, 2021}
\resumeSubHeadingListEnd
\vspace{-5pt}
% }}}
% PRESENTACIONES {{{
\section{~~Presentaciones}
\resumeSubHeadingListStart
\resumeSubItem{Limits and potential of in silico target prediction by chemical similarity}{Poster presentado en el marco del congreso internacional ISCB-LA (Octubre, 2018)}
\vspace{2pt}
\resumeSubItem{A computational chemogenomics method for the prediction of off-target interactions with coagulation factor Xa}{Poster presentado en el marco del congreso internacional EHA (Agosto, 2020)}
\vspace{2pt}
\resumeSubItem{DDTNBI: de novo target prediction using a social network-derived method}{Poster presentado en el marco del congreso internacional ISMB/ECCB (Agosto, 2021)}
\vspace{2pt}
\resumeSubItem{Antifungal drug discovery by chemical similarity-guided network-based inference}{Poster presentado en el marco de la reunion anual de la Sociedad Chilena de Bioinformática (Enero, 2022)}
\resumeSubHeadingListEnd
\vspace{-5pt}
% }}}
% PREMIOS {{{
\section{~~Premios}
\resumeSubHeadingListStart
\resumeSubItem{Concurso de Investigación de pregrado - Invierno 2017}{Proyecto ganador titulado: ''Predicción y priorización \textit{in silico} de nuevos blancos de fármacos.''}
\vspace{2pt}
\resumeSubItem{Concurso de Investigación de pregrado - Verano 2020}{Proyecto ganador titulado: ''Utilización de redes bioquímicas para la predicción de nuevos fármacos para el factor Xa de coagulación (FXa)''}
\vspace{2pt}
\resumeSubHeadingListEnd
% }}}
% OTROS {{{
\section{~~Otros}
\resumeSubHeadingListStart
\resumeSubItem{Codelegado de la Asociación de Estudiantes de Bioquímica (ANEB)}{Periodo: 2018}
\vspace{2pt}
\resumeSubItem{Miembro de la Asociación de Estudiantes de Bioquímica (ANEB)}{Periodo: 2016 - presente}
\vspace{2pt}
\resumeSubItem{Miembro de la International Society for Computational Biology (ISCM)}{Periodo: 2018 y 2021}
\vspace{2pt}
\resumeSubHeadingListEnd
% }}}
\end{document}
